\chapter{Le deuxième}
\label{chapitre:deux}

\lipsum[1-2]

\section{Environnements mathématiques}


\begin{definition}
  Une définition est une séquence finie de mots.
\end{definition}
\lipsum[2]
\begin{definition}
  Une définition est aussi
\end{definition}
\lipsum[2]

\begin{proposition}
Lorem ipsum dolor sit amet, consectetuer adipiscing elit
\end{proposition}
\lipsum[3-4]

\begin{lemma}
Lorem ipsum dolor sit amet, consectetuer elit
\end{lemma}
\begin{proof}
\lipsum[2]
\end{proof}

\begin{theorem}[Théorème général]
Ut purus elit, vestibulum ut, placerat ac, adipiscing vitae, felis.
\end{theorem}
\begin{proof}
\lipsum[3]
\end{proof}
\begin{corollary}
Curabitur dictum gravida mauris.
\end{corollary}
\begin{proof}
\lipsum[4]
\end{proof}
\begin{corollary}
Nam arcu libero, nonummy eget, consectetuer id, vulputate a, magna.
\end{corollary}
\begin{proof}
\lipsum[4]
\end{proof}

\section{Utilisation d'un index}

Il peut être utile de définir un \gls{glossaire} ou un index de \glspl{terme} importants de façon à en indexer l'usage à la fin du document.
Cet exemple utilise le paquetage \href{https://en.wikibooks.org/wiki/LaTeX/Glossary}{\texttt{glossaries}}, qui permet (optionnellement) de fournir des définitions et indexer les usages de termes.
Il est possible d'utiliser d'autre paquetages, comme \href{https://en.wikibooks.org/wiki/LaTeX/Indexing#Using_makeidx}{\texttt{makeidx}}, qui ne font qu'indexer.

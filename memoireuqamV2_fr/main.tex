\documentclass[11pt]{memoireuqamV2}

%%%%%%%%%%%%%%%%%%%%%%%%%%%%%%%%%%%
% OVERLEAF: Utiliser TexLive 2021 %
%%%%%%%%%%%%%%%%%%%%%%%%%%%%%%%%%%%

% \usepackage{natbib}  % optionnel, doit être chargé avant babel
% \renewcommand{\cite}[1]{\citep{#1}}  % redéfinition utile

\usepackage[french]{babel}
\usepackage[utf8x]{inputenc} % Pour utiliser les caractères accentués
\usepackage[T1]{fontenc}
\usepackage{url}
\usepackage[a-1b]{pdfx}
\usepackage{hyperref}
\usepackage{lipsum}
\usepackage{layout}
\makeatletter
\renewcommand*{\lay@value}[2]{%
  \strip@pt\dimexpr0.351459\dimexpr\csname#2\endcsname\relax\relax mm%
}
\makeatother

% optionnel
\usepackage{graphicx}% Pour les figures
\usepackage{amsfonts}
\usepackage{standalone}
\usepackage{tikz}
\usepackage{pgfplots}

% police similaire à calibri, optionnelle (voir https://guidemt.uqam.ca/gabarit-et-mise-en-page/#ressource-697)
\usepackage[sfdefault,lf]{carlito}
% \usepackage{times}


%%%%%%%%%%%%
% Acronymes et symboles
%%%%%%%%%%%%
\usepackage[toc,acronym]{glossaries}
\defglsentryfmt[main]{\textbf{\glsgenentryfmt}}
\newglossary[slg]{symbols}{syi}{syg}{Notation}
\loadglsentries{glossaires/notation}
\loadglsentries{glossaires/acronymes}
\loadglsentries{glossaires/glossaire}
\makenoidxglossaries

\begin{document}%\layout

%%%%%%%%%%%%%%%%%%%%
% Pour la page titre
%%%%%%%%%%%%%%%%%%%%
\title{Mon titre}
% Votre nom complet tel qu'il apparaît à votre dossier du registrariat de l'UQAM
\author{Mon nom}
% Année et mois courant sauf si spécifié autrement pas \degreemonth et \degreeyear
%\degreemonth{mois du dépôt}
%\degreeyear{année du dépôt}
\uqammemoire %% ou \uqamthese ou \uqamrapport ou \uqamprop 
\matiere{mathématiques}


\pagenumbering{roman} % numérotation des pages en chiffres romains
\thispagestyle{empty}        % La page titre n'est pas numérotée
\maketitle
\stepcounter{page}

%%%%%%%%%%%%%%%%%%%%
% Page préliminaires
%%%%%%%%%%%%%%%%%%%%

\chapter*{Remerciements}
Merci à toutes et tous.
\tableofcontents % Pour générer la table des matières
\listoffigures% Pour générer la liste des figures
\listoftables % Pour générer la liste des tableaux

% Ici, on suppose une liste d'acronymes et une fiche de notation. Supprimez ces lignes si elles sont sans objet.
\glsaddall[types={acronym}]
\printnoidxglossary[type=acronym,toctitle=ACRONYMES,nonumberlist,sort=def]
\glsaddall[types={symbols}]
\printnoidxglossary[type=symbols,style=tree,nonumberlist,sort=def,toctitle=NOTATION]
\glsresetall

\begin{abstract}
    \lipsum[1-4]
\end{abstract}
% Utilisez l'environnement  abstract pour rédiger votre résumé

%%%%%%%%%%%%%%%%%%%%
% Document principal
%%%%%%%%%%%%%%%%%%%%
\pagenumbering{arabic}

% Utilisez l'environnement  introduction pour rédiger votre introduction
\begin{introduction}
\label{chapitre:introduction}


\lipsum[1-8]

\end{introduction}

\chapter{Le premier}
\label{chapitre:un}

Le texte peut être découpé en sections, sous-sections, etc. On peut s'y référer en leur attribuant un label, e.g.\verb|\label{identifiant_de_ma_section}|, et s'y référant plus tard par \verb|\ref{identifiant_de_ma_section}|.
Cela évite d'avoir à entretenir manuellement les numéros de section si les choses sont déplacées.
Je peux ainsi référer à la Section ~\ref{sec:secetss}.

\section{Section : Sections et sous-sections}
\label{sec:secetss}
\lipsum[2]

\subsection{Sous-section}
\label{sec:masoussec}
\lipsum[3]
\subsubsection{Sous-sous-section}
\label{sec:massoussoussec}

\lipsum[1-2]

\paragraph{Paragraphe}
\label{sec:unparagraphe}
\lipsum[3-4]

\section{Figures, tableaux, notes de bas de page}

\begin{figure}
    \centering
    % Titre au-dessus de la figure
    % \caption{Une figure}
    \includestandalone{figures/schema1}
    \caption{Une figure}
    \label{fig:figure1}
\end{figure}

\begin{table}
    \centering
    \begin{tabular}{cc}
        \hline
        modèle & valeur \\
        \hline
         A & 1.0 \\
         A & 1.0 \\
         A & 1.0 \\
         A & 1.0 \\
         A & 1.0 \\
        \hline
    \end{tabular}
    \caption{Un autre tableau}
    \label{tab:tableau2}
\end{table}

Les figures et tableaux sont insérées par les environnement standards \verb|figure| et \verb|table|.
Les notes de bas de page sont insérées avec la commande \verb|\footnote|.
Nous faisons référence au Tableau~\ref{tab:tableau1}\footnote{Le placement requis des titres de figures et tableaux peut varier. Consultez votre département d'attache.}.
Voici une autre phrase nécessitant une note de bas de page\footnote{\lipsum[1]}.

\lipsum[4-6]

\begin{table}
    \centering 
    % Titre au-dessus du tableau
    % \caption[Titre court pour liste des tableaux]{\lipsum[1]}
    \begin{tabular}{lccc}
        \hline
        région & modèle & coût & valeur \\
        \hline
         Première & A & 1 & 1.0\\
          & B & 1 & 1.0\\
         Deuxième & C & 1 & 1.0\\
          & D & 1 & 1.0\\
         & E & 2 & 5.0 \\
        \hline
         & Autre & 2 & 5.0 \\
        \hline
    \end{tabular}
    \caption[Titre court pour liste des tableaux]{\lipsum[1]}
    \label{tab:tableau1}
\end{table}


\begin{figure}
    \centering
    \includestandalone{figures/graphique1}
    \caption[Titre court pour la liste des figures]{Longue explication :  Lorem ipsum dolor sit amet, consectetuer adipiscing elit. Ut purus elit, vestibulum ut, placerat ac, adipiscing vitae, felis. Curabitur dictum gravida mauris. Nam arcu libero, nonummy eget, consectetuer id, vulputate a, magna. Donec vehicula augue eu neque. }
    \label{fig:figure2}
\end{figure}



\section{Citer des références bibliographiques}


Citons quelques références indispensables à la compréhension des travaux exposés ci-après~\cite{gingras1991institutionnalisation,glardon2010analyse}.
Les explications de certains experts~\cite{mcneil2019grand, messier1989influence} seront également nécessaires.
\chapter{Le deuxième}
\label{chapitre:deux}

\lipsum[1-2]

\section{Environnements mathématiques}


\begin{definition}
  Une définition est une séquence finie de mots.
\end{definition}
\lipsum[2]
\begin{definition}
  Une définition est aussi
\end{definition}
\lipsum[2]

\begin{proposition}
Lorem ipsum dolor sit amet, consectetuer adipiscing elit
\end{proposition}
\lipsum[3-4]

\begin{lemma}
Lorem ipsum dolor sit amet, consectetuer elit
\end{lemma}
\begin{proof}
\lipsum[2]
\end{proof}

\begin{theorem}[Théorème général]
Ut purus elit, vestibulum ut, placerat ac, adipiscing vitae, felis.
\end{theorem}
\begin{proof}
\lipsum[3]
\end{proof}
\begin{corollary}
Curabitur dictum gravida mauris.
\end{corollary}
\begin{proof}
\lipsum[4]
\end{proof}
\begin{corollary}
Nam arcu libero, nonummy eget, consectetuer id, vulputate a, magna.
\end{corollary}
\begin{proof}
\lipsum[4]
\end{proof}

\section{Utilisation d'un index}

Il peut être utile de définir un \gls{glossaire} ou un index de \glspl{terme} importants de façon à en indexer l'usage à la fin du document.
Cet exemple utilise le paquetage \href{https://en.wikibooks.org/wiki/LaTeX/Glossary}{\texttt{glossaries}}, qui permet (optionnellement) de fournir des définitions et indexer les usages de termes.
Il est possible d'utiliser d'autre paquetages, comme \href{https://en.wikibooks.org/wiki/LaTeX/Indexing#Using_makeidx}{\texttt{makeidx}}, qui ne font qu'indexer.


% Utilisez l'environnement  conclusion pour rédiger votre conclusion
\begin{conclusion}
\label{chapitre:conclusion}

\lipsum[1-4]

\end{conclusion}

%%%%%%%%%%%%%%%%%%%%
% Page liminaires
%%%%%%%%%%%%%%%%%%%%
\appendix % À partir de cette ligne toutes les commandes \chapter subséquentes seront des appendices
\chapter{Une annexe}
\label{annexe:un}

\lipsum[1-4]

% index
\printnoidxglossary[toctitle=INDEX,title=INDEX]

\bibliographystyle{apalike-uqam}
\bibliography{references}
\end{document}
